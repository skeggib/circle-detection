\documentclass{article}

\title{Détection de cercles}
\author{Sébastien Klasa}
\date{3 janvier 2019}

\begin{document}

\maketitle

\section{Exercice 1}

\paragraph{Question 1} Pour $\delta r = 2$ nous aurons $\left\lfloor \frac{r_{max} - r_{min}}{\delta r} \right\rfloor = 49$ valeurs discrètes. Pour $\delta r = 0.5$ nous aurons $198$ valeurs.

\paragraph{Question 2} Nous pouvons décrire $\left\lfloor \frac{r_{max} - r_{min}}{\delta r} \right\rfloor \times \left\lfloor \frac{c_{max} - c_{min}}{\delta c} \right\rfloor \times \left\lfloor \frac{rad_{max} - rad_{min}}{\delta rad} \right\rfloor = 1332936$ cercles avec ces trois variables.

\paragraph{Question 3} $acc(1,1,1)$ correspond au cercle de rayon 1 centré en $(1,1)$. $acc(10,7,30)$ correspond au cercle de rayon 30 centré en $(10,7)$.

\paragraph{Question 4} La case associée est $acc(40,40,13)$.

\end{document}